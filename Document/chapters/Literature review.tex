
\chapter{Literature review}
Recent advancements in Phase-Locked Loop (PLL) design have increasingly incorporated Time-to-Digital Converters (TDCs) as replacements for traditional analog phase detectors. This shift enables higher resolution,
better scalability in advanced CMOS nodes, and improved noise performance. This literature review examines key contributions in the field of TDCs from recent research, providing an overview on second
and third generation TDC architectures, and focusing on bipolar TDCs for DPLL applications.

Before delving into specific designs, it is important to differentiate between the three generations of TDCs:
\begin{itemize}
    \item First Generation: According to Henzler \cite{Henzler2010}, these are analog TDCs that convert the time interval into a voltage to later digitize it with a traditional analog-to-digital converter (ADC). 
    The main drawbacks of this approach are the limited dynamic range because of the trade off between resolution and acquisition range, and the susceptibility to noise and process variations.
    \item Second Generation: Those TDC that are bounded by a CMOS gate delay, such as the Vernier TDC, the Delay Line TDC, the Time Amplifier TDC, and the ring oscillator TDCs. These TDCs are more robust to noise
    and process variations than first generation TDCs, but they still face challenges in achieving high resolution and wide dynamic range simultaneously.
    \item Third Generation: These TDCs overcome technological limitation of time resolution \cite{EL-Hadbi_TDC_review_2019}. Examples of this generation are the vernier TDC, multi-quantization TDCs and TDC with 
    interpolation schemes.
\end{itemize} 

The research field has approached the design of second and third generation TDCs through several methods, hence the variety of TDC architectures. This shows that there is not a single time-to-digital converter
architecture that is suited and optimized for all applications. In fact the TDC architecture must be optimized for the specific application requirements. For example, a TDC for an All Digital Phase-Locked-Loop (ADPLL)
optimizing for high resolution is proposed by X. Meng et. al. in \cite{Meng2025}, achieving a resolution of 390 fs for a 100 MHz reference clock in 28 nm. However, this TDC is not suitable for applications
requiring a wide dynamic range as it only has a maximum readable time interval of 625 ps. On the other hand, a looped TDC optimizing for wide dynamic range is proposed by N. Tetteh et. al. in \cite{Narku-Tetteh2014},
achieving a time interval of 204.8 ns with a resolution of 8.125 ps and consuming 35 mW in 180 nm CMOS process.

Besides resolution and dynamic range, TDC architectures can be optimized for other parameters such as power consumption and linearity. On the latter point, the most linear TDC architectures are based on delay lines
\cite{Henzler2010}. Jung-Chin Lai et. al. proposed a highly linear TDC based on a vernier delay line achieving a DNL of $\pm 0.8$ LSB and an INL of $\pm 2.2$ LSB in 65 nm CMOS \cite{Lai2017}. This design has a
power consumption of 17.5 mW, a resolution of 6.01 ps and a dynamic range of 5.76 ns.


On the topic of bipolar TDCs (this are delay line TDCs that can measure both positive and negative time intervals), there are fewer contributions in the literature. In fact, there is only one bipolar TDC architecture
reported in modern literature as far as the the author is aware. This TDC is proposed by M. Amin in \cite{MohammadAmin2022} and has been developed in 65 nm technology. The architecture is a 7-bit bipolar TDC based
on two 6-bit unipolar cascaded stages with time amplifiers (TA). The TDC achieves a resolution of 1.7 ps and a dynamic range of $\pm 220$ ps, it also has a DNL of 1.03 LSB and an INL of 1.33 LSB.

Table \ref{tab:art_summary} summarizes key metrics from state-of-the-art implementations:

{\scriptsize
\begin{tabular*}{1\linewidth}{@{\extracolsep{\fill}}ccccccc}
    \hline
    Reference & Method & technology & Resolution & Dynamic Range & DNL & INL \\
    \hline
    \cite{Meng2025} & BBPDs & 28 nm & 0.39 ps & 625 ps & - & 0.27 LSB \\
    \cite{Narku-Tetteh2014} & looped TDC & 180 nm & 8.12 ps & 204.8 ns & 0.64 LSB & 1.21 LSB \\
    \cite{Lai2017} & Vernier RO & 65 nm & 6.01 ps & 5.76 ns & 0.8 LSB & 2.2 LSB \\
    \cite{MohammadAmin2022} & bipolar TA & 65 nm & 1.7 ps & 220 ps & 1.03 & 1.33 LSB \\
\end{tabular*}
}
\captionof{table}{Summary of state-of-the-art TDC implementations.}
\label{tab:art_summary}


While significant progress has been made in TDC resolution, dynamic range and power consumption, a need remains for a low-complexity, robust bipolar TDC than can be used as a phase-frequency detector
for an ADPLL that is capable of providing a wide linear range (from $-2\pi$ to $+2\pi$, i.e. approximately $\pm$ 10 ns for a 100 MHz reference clock), thus a wide lock-in range \cite{Meng2025}. It is the main
aim of this thesis to address this gap by proposing a novel bipolar TDC architecture that meets these requirements.

\noindent\rule{\textwidth}{1pt}