\chapter{Theoretical framework}

\section{Phase-locked loop fundamentals}
\subsection{Basic structure}
\subsection{Key PLL parameters}
\subsubsection{Phase noise / jitter}
Jitter is defined as the time deviation ($\Delta_t$) of a signal's transition edges from their ideal positions in time. It is a metric of the outmost importance in the design of PLLs 
as it is a direct measure of the quality of the clock signal generated. Phase noise describes the same phenomenon in the frequency domain (the phase noise of a signal is the Fourier 
transform of the jitter) and is usually expressed in dBc/Hz.
\subsubsection{Output frequency}
It is defined as the range of frequencies that the PLL is capable of generating and can be determined by the VCO output range and the division ratio of the feedback frequency divider. This
is a key metric in establishing the application of the PLL (e.g., clock generation or RF synthesizer) and it bears significant importance in the design process due to the tradeoff 
it has with the phase noise performance of the PLL.
\subsubsection{Loop bandwidth}
The closed-loop bandwidth of a PLL is the frequency range over which the PLL can track the phase/frequency variations of the input signal (from DC to -3 dB from the open-loop 
gain). It affects the acquisition time and phase noise performance of the PLL.
\subsubsection{Noise bandwidth}
dfbdfb
\subsubsection{Lock-in time}

\subsubsection{Pull-in time}
danan
\subsubsection{Lock-in range}
adnna
\subsubsection{Pull-in range}
anan
\subsubsection{Pull-out range}
nana
\subsubsection{Hold range}
fdan
\subsubsection{SNR}
adfn
\subsubsection{Power consumption}
dfanna
\subsubsection{Spurious tones}
fdafdhdhdah

\subsection{Analog phase-locked loops}
dfanadn
\subsection{Linearized model}
daan
\subsection{Digital phase-locked loops}
anan
\section{Time-to-digital converters}
\subsection{Delay-locked loop fundamentals}
\subsection{TDC as a phase detector}

\section{65 nm CMOS technology}
