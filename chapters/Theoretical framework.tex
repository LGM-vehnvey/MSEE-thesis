\chapter{Theoretical framework}

\section{Phase-locked loop fundamentals}
\subsection{Basic structure}
\subsection{Key PLL parameters}
\subsubsection{Phase noise / jitter}
sggs
\subsubsection{Output frequency}
It is defined as the range of frequencies that the PLL is capable of generating and can be determined by the VCO output range and the division ratio of the feedback frequency divider. This
is a key metric in establishing the application of the PLL (e.g., clock generation or RF synthesizer) and it bears significant importance in the design process due to the tradeoff 
it has with the phase noise performance of the PLL.
\subsubsection{Loop bandwidth}
\subsubsection{Noise bandwidth}
dfbdfb
\subsubsection{Lock-in time}
dnsds
\subsubsection{Pull-in time}
danan
\subsubsection{Lock-in range}
adnna
\subsubsection{Pull-in range}
anan
\subsubsection{Pull-out range}
nana
\subsubsection{Hold range}
fdan
\subsubsection{SNR}
adfn
\subsubsection{Power consumption}
dfanna
\subsubsection{Spurious tones}
fdafdhdhdah

\subsection{Analog phase-locked loops}
dfanadn
\subsection{Linearized model}
daan
\subsection{Digital phase-locked loops}
anan
\section{Time-to-digital converters}
\subsection{Delay-locked loop fundamentals}
\subsection{TDC as a phase detector}

\section{65 nm CMOS technology}
